\documentclass[10pt,a4paper,oneside]{article}
\usepackage[utf8]{inputenc}
\usepackage{amsmath}
\usepackage{amsfonts}
\usepackage{amssymb}
\usepackage{graphicx}
\usepackage[]{placeins}
\usepackage[parfill]{parskip}
\usepackage[left=2cm,right=2cm,top=2cm,bottom=2cm]{geometry}
\author{Julian Bauer}
\title{Feedback: Extension proposal of mechkit.material.Isotropic}
\begin{document}
\maketitle
\sloppy

\section{Erweiterungsvorschlag}
Sind zwei Komponenten des Steifigkeitstensors 
\begin{align}
\mathbb{C}
= 
	\begin{bmatrix}
  C_{11} & C_{12} & C_{13} & C_{14} & C_{15} & C_{16} \\
  C_{21} & C_{22} & C_{23} & C_{24} & C_{25} & C_{26} \\
  C_{31} & C_{32} & C_{33} & C_{34} & C_{35} & C_{36} \\
  C_{41} & C_{42} & C_{43} & C_{44} & C_{45} & C_{46} \\
  C_{51} & C_{52} & C_{53} & C_{54} & C_{55} & C_{56} \\
  C_{61} & C_{62} & C_{63} & C_{64} & C_{65} & C_{66}
	\end{bmatrix}_{[\text{Voigt}]}  	\label{eq:stiffness}
\end{align}
mit unterschiedlichen Werten sinnvolle Eingabeparameter zur Erstellung einer Instanz der Klasse mechkit.material.Isotropic?

\section{Analyse}
Das Materialgesetz der linearen Elastizität
\begin{align}
\boldsymbol{\sigma}
=
\mathbb{C}
\left[
	\boldsymbol{\varepsilon}
\right]
\end{align}
mit
\begin{table}[!h]
\centering
	\begin{tabular}{l|l}
		$\boldsymbol{\sigma}$ 		& Cauchy Spannung (Tensor 2. Stufe)\\
		$\boldsymbol{\varepsilon}$ 	& Inf. Verzerrungstensor (Tensor 2. Stufe)\\
		$\mathbb{C}$				& Elastizitätstensor (Tensor 4. Stufe)
	\end{tabular}
\end{table}
\FloatBarrier
kann im Falle der Isotropie als 
\begin{align}
\boldsymbol{\sigma}
=
2 \mu \mathbb{I}^{\text{S}} \boldsymbol{\varepsilon}
+
\lambda \mathbf{I} \otimes \mathbf{I}
\end{align}
mit
\begin{table}[!h]
\centering
	\begin{tabular}{l|l}
		$\mathbb{I}^{\text{S}}$ & Identität auf symmetrischen Tensoren 4. Stufe (Tensor 4. Stufe)\\
		$\mathbf{I}$ 			& Identität auf Tensoren 2. Stufe (Tensor 2. Stufe)\\
		$\otimes$				& Dydisches Produkt\\
		$\mu$					& Schubmodul, 2. Lame'scher Parameter\\
		$\lambda$				& 1. Lame'scher Parameter
	\end{tabular}
\end{table}
\FloatBarrier
notiert werden.

Es gilt
\begin{align}
\mathbb{I}^{\text{S}}
&=
\frac{1}{2}
\left(
	\delta_{ik} \delta_{lj}+ \delta_{il} \delta_{kj}
\right)
\mathbf{e}_{i}
\otimes
\mathbf{e}_{j}
\otimes
\mathbf{e}_{k}
\otimes
\mathbf{e}_{l} \\
&=
	\begin{bmatrix}
  1 & 0 & 0 & 0 & 0 & 0 \\
  0 & 1 & 0 & 0 & 0 & 0 \\
  0 & 0 & 1 & 0 & 0 & 0 \\
  0 & 0 & 0 & 1 & 0 & 0 \\
  0 & 0 & 0 & 0 & 1 & 0 \\
  0 & 0 & 0 & 0 & 0 & 1 \\
	\end{bmatrix}_{[\text{Mandel6}]}\\
&=
	\begin{bmatrix}
  1 & 0 & 0 & 0 & 0 & 0 \\
  0 & 1 & 0 & 0 & 0 & 0 \\
  0 & 0 & 1 & 0 & 0 & 0 \\
  0 & 0 & 0 & \frac{1}{2} & 0 & 0 \\
  0 & 0 & 0 & 0 & \frac{1}{2} & 0 \\
  0 & 0 & 0 & 0 & 0 & \frac{1}{2} \\
	\end{bmatrix}_{[\text{Voigt}]}
\end{align}
und
\begin{align}
\mathbf{I} \otimes \mathbf{I}
&=
\delta_{ij} \delta_{kl}
\mathbf{e}_{i}
\otimes
\mathbf{e}_{j}
\otimes
\mathbf{e}_{k}
\otimes
\mathbf{e}_{l} \\
&=
	\begin{bmatrix}
  1 & 1 & 1 & 0 & 0 & 0 \\
  1 & 1 & 1 & 0 & 0 & 0 \\
  1 & 1 & 1 & 0 & 0 & 0 \\
  0 & 0 & 0 & 0 & 0 & 0 \\
  0 & 0 & 0 & 0 & 0 & 0 \\
  0 & 0 & 0 & 0 & 0 & 0 \\
	\end{bmatrix}_{[\text{Mandel6}]}\\
&=
	\begin{bmatrix}
  1 & 1 & 1 & 0 & 0 & 0 \\
  1 & 1 & 1 & 0 & 0 & 0 \\
  1 & 1 & 1 & 0 & 0 & 0 \\
  0 & 0 & 0 & 0 & 0 & 0 \\
  0 & 0 & 0 & 0 & 0 & 0 \\
  0 & 0 & 0 & 0 & 0 & 0 \\
	\end{bmatrix}_{[\text{Voigt}]}
\end{align}
mit
\begin{table}[!h]
\centering
	\begin{tabular}{l|l}%
		$\begin{bmatrix}.&.\\.&.\end{bmatrix}_{\text{[Mandel6]}}$ & Komponentendarstellung in Mandel6 Notation \\
		$\begin{bmatrix}.&.\\.&.\end{bmatrix}_{\text{[Voigt]}}$ & Komponentendarstellung in Voigt Notation.
	\end{tabular}
\end{table}
\FloatBarrier

Daraus folgt für die Komponenten des Steifigkeitstensors (vgl. Gleichung \ref{eq:stiffness})
\begin{align}
C_{11} = C_{22} = C_{33} = 2\mu + \lambda	\\
C_{12} = C_{13} = C_{21} = C_{23} = C_{31} = C_{32}
\end{align}

\section{Diskussion}


\end{document}